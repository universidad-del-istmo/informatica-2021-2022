%%%%%%%%%%%%%%%%%%%%%%%%%%%%%%%%%%%%%%%%%
% Programming/Coding Assignment
% LaTeX Template
%
% This template has been downloaded from:
% http://www.latextemplates.com
%
% Original author:
% Ted Pavlic (http://www.tedpavlic.com)
%
% Note:
% The \lipsum[#] commands throughout this template generate dummy  text
% to fill the template out. These commands should all be removed when 
% writing assignment content.
%
% This template uses a Perl script as an example snippet of code, most other
% languages are also usable. Configure them in the "CODE INCLUSION 
% CONFIGURATION" section.
%
%%%%%%%%%%%%%%%%%%%%%%%%%%%%%%%%%%%%%%%%%

%----------------------------------------------------------------------------------------
%	PACKAGES AND OTHER DOCUMENT CONFIGURATIONS
%----------------------------------------------------------------------------------------

\documentclass{article}

\usepackage{fancyhdr} % Required for custom headers
%\usepackage{lastpage} % Required to determine the last page for the footer
\usepackage{graphicx} % Required to insert images
\usepackage{color}
\usepackage{listings} % Required for insertion of code
\usepackage{courier} % Required for the courier font
\usepackage{multirow}
\usepackage{hyperref}


% Margins
\topmargin=-0.45in
\evensidemargin=0in
\oddsidemargin=0in
\textwidth=6.5in
\textheight=9.0in
\headsep=0.25in

\linespread{1.1} % Line spacing

%----------------------------------------------------------------------------------------
%	CODE INCLUSION CONFIGURATION
%----------------------------------------------------------------------------------------

\definecolor{MyDarkGreen}{rgb}{0.0,0.4,0.0} % This is the color used for comments
\lstloadlanguages{c} % Load Perl syntax for listings, for a list of other languages supported see: ftp://ftp.tex.ac.uk/tex-archive/macros/latex/contrib/listings/listings.pdf
\lstset{language=[sharp]c, % Use Perl in this example
        frame=single, % Single frame around code
        basicstyle=\small\ttfamily, % Use small true type font
        keywordstyle=[1]\color{Blue}\bf, % Perl functions bold and blue
        keywordstyle=[2]\color{Purple}, % Perl function arguments purple
        keywordstyle=[3]\color{Blue}\underbar, % Custom functions underlined and blue
        identifierstyle=, % Nothing special about identifiers                                         
        commentstyle=\usefont{T1}{pcr}{m}{sl}\color{MyDarkGreen}\small, % Comments small dark green courier font
        stringstyle=\color{Purple}, % Strings are purple
        showstringspaces=false, % Don't put marks in string spaces
        tabsize=5, % 5 spaces per tab
        %
        % Put standard Perl functions not included in the default language here
        morekeywords={rand},
        %
        % Put Perl function parameters here
        morekeywords=[2]{on, off, interp},
        %
        % Put user defined functions here
        morekeywords=[3]{test},
       	%
        morecomment=[l][\color{Blue}]{...}, % Line continuation (...) like blue comment
        numbers=left, % Line numbers on left
        firstnumber=1, % Line numbers start with line 1
        numberstyle=\tiny\color{Blue}, % Line numbers are blue and small
        stepnumber=5 % Line numbers go in steps of 5
}

\newcommand{\horrule}[1]{\rule{\linewidth}{#1}}

% Creates a new command to include a perl script, the first parameter is the filename of the script (without .pl), the second parameter is the caption
\newcommand{\perlscript}[2]{
\begin{itemize}
\item[]\lstinputlisting[caption=#2,label=#1]{#1.cs}
\end{itemize}
}

\begin{document}

\begin{tabular}{l l}
\multirow{5}{*}{\includegraphics[width=2cm]{../../recursos/logo.png}} & Universidad del Istmo de Guatemala \\
 & Facultad de Ingenieria \\
 & Ing. en Sistemas y Ciencias de la Computaci\'on \\
 & Informatica 1 \\
 & Prof. Ernesto Rodriguez - \href{mailto:erodriguez@unis.edu.gt}{erodriguez@unis.edu.gt} \\
\end{tabular}
\\\\\\

\begin{center}
        \horrule{0.5pt}
        \huge{Laboratorio \#1} \\
        \large{Fecha de entrega: 06 de Agosto, 2021 - 11:59pm} \\
        \large{Modalidad de trabajo: Individual o Parejas}
        \horrule{1pt}
\end{center}

\emph{Instrucciones: Resolver los problemas que se le presentan a
continuaci\'on. Este trabajo debe ser entregado como un pull request
en Github. Instrucciones e informaci\'on acerca de un pull request
se encuentran al final de este documento y tambien se describiran
en clase.}

% \perlscript{homework_example}{Sample Perl Script With Highlighting}

\section*{Ejercicio \#1 (50\%): Multiplicaci\'on Inductiva}
De una \emph{definici\'on inductiva} para multiplicar dos \emph{numeros de peano}.
Tiene permitido utilizar la definicion de suma que se estudio en clase en
su definici\'on de multiplicaci\'on. Esta se presenta a continuaci\'on:
\[
    \begin{array}{l}
        n \oplus 0 = n \\
        0 \oplus m = m \\
        n \oplus \mathbf{s}(a) = \mathbf{s}(n \oplus a)
    \end{array}
\]
Recuerde que una multiplicaci\'on es una sucessi\'on de sumas. Utilize
este conocimiento para representar dicha succesi\'on de forma inductiva.
Por ejemplo: $3 \otimes 4 = 3 \oplus 3 \oplus 3 \oplus 3 = 4 \oplus 4 \oplus 4$.

\section*{Ejercicio \#2: Inducci\'on (50\%)}

Utilize el \emph{principio de inducci\'on} para demostrar que:
\[
    a \oplus (b \oplus c) = (a \oplus b) \oplus c 
\]
En donde $a, b, c$ son \emph{numeros de peano} y $\oplus$ es la
suma de numeros de peano estudiada en clase.

\section*{Entrega}

\begin{enumerate}
\item{Crear una cuenta en \url{github.com}}
\item{\href{https://github.com/git-guides/install-git}{Instalar git} en su computadora.}
\item{Navegar al \emph{repositorio del curso}: \url{https://github.com/universidad-del-istmo/informatica-2021-2022}}
\item{Hacer un \href{https://docs.github.com/en/get-started/quickstart/fork-a-repo}{fork} del
repositorio presionando el boton de fork.}
\item{Navegar a la copia del repositorio creada mediante fork.}
\item{\href{https://docs.github.com/en/github/creating-cloning-and-archiving-repositories/cloning-a-repository-from-github/cloning-a-repository}{Clonar} el
repositorio creado a su computadora.}
\item{Crear una \href{https://docs.github.com/en/github/collaborating-with-pull-requests/proposing-changes-to-your-work-with-pull-requests/about-branches}{rama} en
la copia en su computadora de su repositorio mediante ``git checkout -b laboratorio1''. Esta
rama permitira trabajar en este laboratorio de forma aislada.}
\item{En el repositorio clonado, crear una \emph{carpeta de entrega} ubicada en ``Informatica I\textbackslash laboratorios\textbackslash laboratorio 1\textbackslash [Nombre del grupo]''}
\item{Crear un archivo llamado ``grupo.txt'' en su \emph{carpeta de entrega} y apuntar los nombres
de los alumnos que elaboraron ese trabajo.}
\item{Colocar su trabajo en la \emph{carpeta de entrega}.}
\item{Crear una nueva revisi\'on del repositorio mediante \href{https://github.com/git-guides/git-commit}{git commit}.}
\item{Empujar la nueva revisi\'on a su copia del repositorio mediante \href{https://github.com/git-guides/git-push}{git push}.}
\item{Crear un \href{https://docs.github.com/en/github/collaborating-with-pull-requests/proposing-changes-to-your-work-with-pull-requests/about-pull-requests}{pull request} con sus cambios en el \emph{repositorio del curso}.
Asegurese de seleccionar la rama correcta de su repositorio y selecionar \emph{master} como
rama del repositorio del curso.}
\end{enumerate}
\end{document}